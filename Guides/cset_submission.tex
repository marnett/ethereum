\documentclass[10pt,twocolumn,letterpaper]{article}
\usepackage[margin=1in]{geometry}
\usepackage{listings}
\usepackage{framed}
\usepackage{enumitem}
\usepackage[small]{titlesec}
\usepackage[usenames,dvipsnames,svgnames,table]{xcolor}
\usepackage[framemethod=tikz]{mdframed}
\usepackage{color}
\usepackage{minted}
\PassOptionsToPackage{hyphens}{url}\usepackage{hyperref}

\newcommand{\elaine}[1]{{\color{red}{[elaine: #1]}}}
\newcommand{\ignore}[1]{}

\lstset{
	tabsize = 4
}

\begin{document}
\title{Step by Step towards Programming a Safe Smart Contract}

\date{}
\maketitle

%\setcounter{tocdepth}{5}
%\tableofcontents

\newpage
\section{Introduction}

%\elaine{rewrite}
Completely decentralized crypto-currencies like Bitcoin 
\elaine{cite}
and other altcoins 
\elaine{cite}
have captured the public's attention and interest, 
and have been much more successful than any prior incarnations of electronic
cash. Many would call the rise 
of these electronic currencies a technological revolution, and the ``wave of
the future''.
Emerging altcoins such as Ethereum \elaine{cite} and Counterparty \elaine{cite}
extend Bitcoin's design by offering a rich programming language for 
writing ``smart contracts.'' Smart
contracts are user-defined programs that specify rules 
governing transactions, and that are enforced by
the network of peers (assuming the underlying crypto-currency is secure). 
In comparison with traditional
financial contracts, smart contracts carry the promise of low legal 
and transaction fees, and can potentially
lower the bar of entry for users.

In Fall 2014, at the University of [anonymized], 
we organized a new, hands-on
smart contract programming lab in our undergraduate security
class [course number and name anonymized] -- the first of its
kind that has ever been attempted.


%Cryptocurrencies, including Bitcoin, Ethereum, and many others, are an exciting new technology. They are experimental distributed systems that allow users to manipulate virtual currency. Actual stored wealth and monetary value are at stake! Ethereum is the first embodiment of the more general idea: it provides an expressive and flexible programming environment for controlling and interacting with money.

%This tutorial is intended for instructors
%who wish to conduct a smart  
%contract programming lab, or students/developers
%who want to learn about smart contract programming.

\ignore{
The first part of this lab consists of step-by-step examples illustrating basic design of functional smart contracts. We highly recommend you take a hands-on approach, and interact with these smart contract examples using the Ethereum simulator! The accompanying materials to everything you need to get started with experimenting, including  a virtual machine image, basic instructions, and a language guide.

The second part of this lab focuses on designing smart contracts that achieve their intended goals, and are robust to attacks. 
Although our lab makes us of a simulator, the smart contracts you write can also be used in the live Ethereum network\footnote{At the time of this writing, the only live Ethereum network is a test network, since the main network has not yet launched.} The basic concepts we discuss apply to other cryptocurrencies as well (including Bitcoin), so most of what you learn will be transferable.
}

\paragraph{Smart contract programming: unique challenges.}
Although smart contract programming in many ways resembles 
traditional programming, 
it raises important new security challenges. 
%Smart contract design is inherently security-oriented. 
Contracts are ``play-for-keeps'', since virtual currencies have real value. 
If you load money into a buggy smart contract, you will likely lose it. 
Further, smart contract programming requires
an ``economic thinking'' perspective that traditional
programmers need not have. 
Contracts must be written to ensure fairness even when
counterparties may be incentivized to cheat in arbitrary ways to maximize
their economic gains.

\elaine{say more about all parties being 
selfish, and may do malicious things to maximize its
financial gains.} 

\elaine{stress that even programming a very simple
game like rock paper scissors was difficult
and exposed many problems.}


\paragraph{Open-source course and lab materials.}
As an outcome of our lab, we observed several classes
of typical mistakes students made. 
%While bugs such as buffer overflows are 
%typical of 
In contrast to  
traditional software development where 
bugs such as buffer overflows are typical, 
in our lab, we observed 
bugs and pitfalls that arise due to 
%Several of such mistakes 
%are specific to 
the ``smart contract'' nature of the programs. 
%may not be observa
%These mistakes suggest that programming safe smart contracts 
%is difficult. 

Based on lessons and insights 
drawn through this experimental lab, we have designed
new, open course materials and lab designs 
for smart contract programming.
\elaine{cite anonymously}
We hope that these open-source course materials and labs
will help both instructors who 
wish to teach smart contract programming and students/developers who 
wish to teach themselves smart contract programming.
\elaine{probably the langugage can be better.}

\paragraph{Roadmap.}
In the remainder of this paper, we will first give more background on 
crypto-currency and smart contracts (Section \elaine{refer}). 
We will then detail experiences with our lab (Section \elaine{refer}),  
the typical pitfalls we observed in smart  
contract programming (Section \elaine{refer}), 
and the insights and lessons learned. 



%In a decentralized cryptocurrency like Ethereum, smart contract programs
%are propagated to the entire network, and therefore
%security through obscurity  
%Unlike other hands-on labs in cryptography (e.g., sending encrypted emails with GPG), where actual attacks are unlikely or hard to observe, the attackers in a cryptocurrency are much more apparent. (For example, if you publish a Bitcoin transaction with a ``weak'' brainwallet password, it will be stolen within seconds by hackers who have built tables of the most common passwords. 
%\elaine{Cite Joe's Bonneau's paper}
\ignore{
Smart contract design also requires economic thinking. We use a running example about a rock-paper-scissors game. To help keep incentives in focus, we reward the winner with a monetary prize, so both participants have a stake in the outcome.  Other, more clearly “useful” application include derivative financial instruments, for example that allow people to buy or sell insurance against events that can be “logged” by the network, such as the price of another cryptocurrency. Smart contracts can also be used to raise “crowdfunding” money with a Kickstarter-like assurance contract, that gives contributors a refund if a donations target isn’t reached. In all of these applications, we will want to guarantee that the smart contracts are ``fair'' and aren't profitable to exploit.
}
%% For the users' convenience, we offer
%% a VM image with appropriate  
%% versions of the software pre-installed~\cite{vmimage}.
%% We also provide detailed Ethereum reference manuals
%% geared towards this specific 
%% snapshot of Serpent~\cite{serpentref}.
%% Finally, we also recommend the reader
%% to a more concise, Powerpoint presentation of this tutorial
%% by Elaine Shi and Andrew Miller~\cite{Shi2015}.

%, such that when
%rational miners comprise the majority of compute
%power (or other forms of resources),
%in a Nash equilibrium, it is in the best interest
%of rational miners to honestly execute a
%contract's program logic.

\section{Background}
In this section, we provide some background on cryptocurrencies and the programming model of smart-contracts.

\subsection{The Smart-Contract Programming Model}

\paragraph{The Underlying Cryptocurrency.}
Smart contracts are built on top of an underlying cryptocurrency platform. A cryptocurrency is a decentralized system for interacting with virtual money in a shared global ledger. Users transfer money and interact with contracts by publishing signed messages called \emph{transactions} to the cryptocurrency network. The network consists of nodes (called miners) who propagate information, store data, and update the data by applying transactions. A high-level schematic is shown in Figure~\ref{fig:schematic}.
\elaine{fix}

Although the ideas behind cryptocurrencies date back at least twenty-five years (e.g., cryptographic e-cash~\cite{chaum-ecash}), a recent surge of interest in this technology has been incited by the success of Bitcoin~\cite{bitcoin}. For a comprehensive survey on Bitcoin and other cryptocurrencies, see ~\cite{researchperspectives,bittertobetter}.

We shall make some simplifying assumptions 
about the security model of the underlying cryptocurrency.
Loosely speaking, we assume that the
cryptocurrency has a secure and incentive compatible
consensus protocol. 
In reality, existing decentralized cryptocurrencies
achieve only heuristic security; designing a provably  
secure decentralized consensus protocol under
rationality assumptions is a topic of 
future research.

The main interface provided by the underlying cryptocurrency is an append-only log called a \emph{blockchain}, which imposes a total ordering on transactions submitted by users. The data in the blockchain is guaranteed to be \emph{valid} according to certain predefined rules of the system (e.g., there are no double-spends or invalid signatures). All of the data in the blockchain is public, and every user can access a copy of it. No one can be \emph{prevented} from submitting transactions and getting them included in the blockchain (with at most some small delay). There is global agreement among all nodes and users about the contents of the blockchain,  except for the most recent handful of blocks which have not yet settled.

We also assume that the built-in currency, Ether, has a stable monetary value. Users have an incentive to gain more of (and avoid losing) units of this currency. Anyone can acquire Ether by purchasing or trading for it using other currencies such as Bitcoin or US dollars. The currency is assumed to be fungible; one unit of ether is exactly as valuable as any other, regardless of the currency's ``history.''

\paragraph{Contracts and Addresses.}
The system keeps track of ``ownership'' of the currency by associating each unit of currency to an ``address''. There are two kinds of addresses: one for users, and one for contracts. A user address is a hash of a public key; whoever knows the corresponding private key can spend the money associated to that address. Users can create as many accounts as they want, and the accounts need not be linked to their real identity.

A contract is an instance of a computer program that runs on the blockchain. It consists of program code, a storage file, and an account balance.
Any user can create a contract by posting a transaction to the blockchain.
The program code of a contract is fixed when the contract is created, and cannot be changed.
The contract's code is executed whenever it receives a message, either from a user or from another contract.
While executing its code, the contract may read from or write to its storage file.
A contract can also receive money into its account balance, and send money from its account balance to other contracts or users.

The code of a contract determines how it behaves when it receives messages, under what conditions (and to whom!) it sends  money out, and how it interacts with other contracts by sending messages to them. This document is especially about how to write code for useful and dependable contracts.

\paragraph{Transactions, Messages and Gas.}
A transaction always begins with a message from a user to some recipient address (either another user or a contract). This message must be signed by the user, and can contain data, ether, or both. If the recipient is a contract, then the code of that contract is executed. If that code contains an instruction to send a message to another contract, then that contract's code is executed next. So, a transaction must contain at least one message, but can trigger several messages before it completes.

Messages act a bit like function calls in ordinary programming languages. After a contract finishes processing a message it receives, it can pass a return value back to the sender.

In some cases, a contract can encounter an ``exception'' (e.g., because of an invalid instruction). After an exception, control is also returned to the sender along with a special return code. The state of \emph{all} contract, including account balances and storage, is reverted to what it was just prior to calling the exception-causing message.

Ethereum uses the concept of ``gas'' to discourage overconsumption of resources. The user who creates a transaction must spend some of the ether from that account to purchase gas. During the execution of a transaction, every program instruction consumes some amount of gas. If the gas runs out before the transaction reaches an ordinary stopping point, it is treated as an exception: the state is reverted as though the transaction had no effect, but the ether used to purchase the gas is not refunded! When one contract sends a message to another, the sender can offer only a \emph{portion} of its available gas to the recipient. If the recipient runs out of gas, control returns to the sender, who can use its remaining gas to handle the exception and tidy up.


\subsection{A Taste of Smart-Contract Design}

In this section we give the reader a taste of smart-contract design by describing several Ethereum implementation of simple but useful motivating example. 

Our first example is a financial \emph{swap} instrument that allows two parties, Alice and Bob, to take opposing bets about the price of a stock at some future time. Both parties initially deposit equal amounts of money (as units of Ether currency). After a deadline has passed, the current price of the stock is queried by interacting with a designated stock price authority (which would itself be implemented as a smart contract). Depending on the price at that time, the entire combined deposit is awarded to either Alice or Bob.

\begin{mdframed}
%\small
\begin{minted}[breaklines, tabsize=2, fontsize=\footnotesize, linenos, xleftmargin=10pt]{python}
data Alice, Bob, StockPriceAuthority
data deadline, threshold

# Not shown: collect equal deposits from Alice and Bob

def determine_outcome():
    if block.timestamp > deadline: 
        price = StockPriceAuthority.getPrice()
        if price > threshold:
            send(Alice, self.balance)
        else:
            send(Bob, self.balance)
\end{minted}
\end{mdframed}

This example serves as motivation of the ``useful'' aspects of smart contracts as financial instruments. In our other examples, we will tend to focus on gambling games. It also serves to illustrate several low level aspects of Serpent programming.

The contract is defined using a function \texttt{determine\_outcome}, which any party may invoke.

For our first example, we will make a contract that is normally called "namecoin". It allows for us to create a basic key-value store. A key value store is a data storage structure that allows for us to associate a key with a value, and look up values based on their keys. This contract will have two different functions to call. The first is the key-value registration function and the second is a function that retrieves a value associated with a provided key.

The first function we will look at is $register(key, value)$, which takes a key and value and associates them with each other:

\begin{mdframed}
\begin{minted}[breaklines, tabsize=2, fontsize=\footnotesize, linenos, xleftmargin=10pt]{python}
def register(key, value):
	if self.storage[key] == 0:
		self.storage[key] = value
		return(1)
	else:
		return(-1)
\end{minted}
\end{mdframed}

Let's break this down. This contract essentially consists of an if-else statement. First, we check to see if the key-value is already in storage. We can use the not statement to check if nothing is stored. So if nothing is stored, we will store the value in the persistent key-value store $self.storage[]$. However, what if the key is already taken? We can't just overwrite someone else's key! So, we just return -1. 

\paragraph{Time- and Event-based state transitions.}
In many scenarios, there is a need to adapt the behavior of a contract depending on the the messages it receives, or depending on how much time has passed since a certain event. In other words, several applications need a stateful contract that acts differently to similar messages, depending on its state.

Maintaining the notion of a state in a contract requires a mechanism to handle state transitions, which we classify into event-based and time-based. We present simple approaches for how to express these in Serpent.

In this case, the state changes based on messages that the contract receives. One example is a puzzle contract that gives a reward to the first person who solves a problem, or a game contract that waits for two players to join before starting the game. The behavior of the contract should adapt when such events occur, as otherwise money may be needlessly lost. Such contracts can be straightforwardly implemented by maintaining state variables in the contract storage. The following example is a proof-of-work contract that gives an award to the first message sender who solves a Bitcoin-like proof-of-work puzzle.

%\begin{mdframed}[rightmargin = -1cm, leftmargin = -1cm, linecolor=black, topline=true, bottomline=true,
  %leftline=false, rightline=false, backgroundcolor=lightgray!40]
%\begin{minted}
%[
%frame=lines,
%framesep=2mm,
%baselinestretch=1.2,
%fontsize=\footnotesize,
%linenos
%]
%{python}
\begin{mdframed}
\begin{minted}[breaklines, tabsize=2, fontsize=\footnotesize, linenos, xleftmargin=10pt]{python}
def init(puzzle, target):
     self.storage["isSolved"] = 0                 ## State variable
     self.storage["puzzle"] = puzzle
     self.storage["target"] = target           
 

def receiveSolution(solution):
    if(self.storage["isSolved"] == 0 AND 
    	SHA3([self.storage["puzzle"], solution],2) < self.storage["target"]):
send(msg.sender, 10000)             # Sending reward
self.storage["isSolved"] = 1   # Changing the state variable
\end{minted}
\end{mdframed}

Employing event-based transitions may not be enough to capture all the possible scenarios in typical applications. Think of an auction that accepts any number of bidders, but sets a specific deadline after which no new bids are accepted. The contract in this case should have a way to decide whether to accept bids or not based on the time of the transaction.

There are two simple ways to use refer to the current time in a contract: \texttt{block.timestamp} or \texttt{block.number}. For example, the following is a fragment of an auction contract that only accepts bids submitted before a deadline. The deadline is 100 blocks ahead from the contract creation time.

%\begin{mdframed}[rightmargin = -1cm, leftmargin = -1cm, linecolor=black, topline=true, bottomline=true,
%  leftline=false, rightline=false, backgroundcolor=lightgray!40]
\begin{mdframed}
\begin{minted}[breaklines, tabsize=2, fontsize=\footnotesize, linenos, xleftmargin=10pt]{python}
def init():
     self.storage["deadline"] = 100 + block.number                
     # Think of other auction details  

def receiveBid(bid):
     if(block.number <= self.storage["deadline"]):
                # accept bid
     else:
                # abort
\end{minted}
\end{mdframed}                

Sometimes, complex contracts will need to incorporate both kinds of state transitions. For example, consider a fundraising contract that either concludes immediately after a certain target amount of money is collected, or else after a month passes without reaching the target. Therefore, the contract must change its state if a month passes, or when the contract balance exceeds a threshold, whatever happens first.\\


\section{A Recount of Our Smart Contract Programming Lab}
In our undergraduate security class at the University of [anonymized], 
students are asked to develop smart contract applications of their choice
atop a new cryptocurrency called Ethereum \elaine{cite}.
Ethereum offers a Turing Complete programming language
called Serpent \elaine{cite} for composing smart contracts. 

Students were divided into groups of four.  
Due to the experimental nature of the lab, 
the instructor assigned one of her Ph.D. students 
to closely supervise each group, to ensure that students
obtained hands-on help.

The lab proceeded in two phases. 
The first phase is a 
{\it creation} phase where each group created a 
smart contract application
of their own choice.
The students created a variety of applications, including
\elaine{give a laundry list}.
At the end of the first phase, each group 
made a short presention of their 
application in class.
The instructor, TAs, and students together observed numerous issues  
with the programs that students 
wrote (see Section \elaine{refer} for a detailed
discussion). 

Therefore, we extended the project to a second phase, called 
an {\it amendment} phase. 
The goal of this phase was for students to critique their programs,
find bugs, and amend their designs. 
The instructor and TAs had in-person meetings with each project group
to help them amend their smart contract programs.  
Students also formed pair groups to critique and 
help the other group. 



\section{Pitfalls of Smart Contracts Programming}
%Now that we have gone through and annotated several contract examples it is time to consider a couple key design concepts required to create a secure smart contract. 
%By the end of this section we will talk about several key mistakes that show up in high-level contracts, and you will aim to identify and resolve them in a rock, paper, scissor contract example (RPS).
In this section, we will demonstrate some of the typical pitfalls
we observed for smart contract programming. 
For ease of exposition, we will use
a simple ``Rock, Paper, Scissors''  
example to illustrate three classes of typical mistakes.
Similar mistakes 
were commonly observed in various other 
applications developed by the students.  

\paragraph{Quick overview of our running example.}
We will first give a quick overview of the structure
of our buggy ``Rock, Paper, Scissors'' program, before
we go on to diagnose the bugs.
In this contract, two players 
will play a simple 
``Rock, Paper, Scissors''
for money. 
The contract program consists of three main functions:
\begin{itemize}
\item
{\tt add\_player}:
The players register with the contract
and deposit money to play.
\item
{\tt input}:
Both players input to the contract 
their choice of rock, paper, or scissors.
\item
{\tt check}:
The contract decides a winner
and sends the proceeds to the winner. 
\end{itemize}

As we show below, 
surprisingly, {\it even for a very simple smart contract like this, 
it is difficult 
to make it right}!

\elaine{emphasize this in the intro too.}


\ignore{
In this section, we'll explore the security and incentive alignment pitfalls in designing a smart contract. We'll use an easy-to-understand application as a running example, based on a Rock-Paper-Scissors game. We then analyze a plausible (but subtly buggy) initial implementation, pointing out its flaws. Mistakes resembling these were actually observed in our Smart Contract Programming Lab in ``CMSC 414 - Undergraduate Security''. vThis section is centered around the exercises. We provide hints to guide the reader towards discovering how to improve on them. Our ``reference'' solution can be found in the accompanying materials.
}


\subsection{Errors in Encoding State Machines}

Programming smart contracts typically involves
encoding complex state machines.
Logical errors
in encoding state machines were commonly observed.
The simplest type of logical error is a contract
that leaks money in corner cases.


To illustrate this, let us look at our buggy
``Rock, Paper, Scissors'' example.
Figure~\ref{fig:moneyleaks}
shows the {\tt add\_player} function
where players register with the contract and  
deposit money to play.
The contract would then store the players'
public keys 
and coins deposited \elaine{refer to line numbers}.

This contract has a couple mistakes:
\begin{itemize}[leftmargin=5mm]
\item
First, when a third player sends money to the contract,
the money is leaked (Line \elaine{refer}) 
and the third player does not get a chance
to play the game.
\item
Second, if a player sends an amount of money that is not
exactly 1000 Ethers, \elaine{make sure those names come out in the background section.}
the contract also leaks the money \elaine{refer to line number}.
\end{itemize}

Note that while a player 
can protect itself from the second problem by being defensive, 
{\it it cannot always protect 
itself from the first issue!}
%the second problem can be avoided
%by being a defensive player, the first problem cannot!
In a decentralized cryptocurrency
like Bitcoin or Ethereum, 
multiple parties may be sending inputs 
to the contract simultaneously.
In this case, it is up to the winning miner
(for this block) to decide how to order these transactions.

To fix the bugs in Figure~\ref{fig:moneyleaks}, the contract
should refund the money back to 
a player unless the player is actually registered to play.
A fixed version of our ``Rock, Paper, Scissors'' program is 
available for download at \elaine{fill in anonymous URL.}

\paragraph{More sophisticated logical errors.}
\elaine{write some high level text here.}

\ignore{
The first contract design error we will talk about is contracts causing money to disappear. Some contracts require the participants to send an amount of money to enter the contract (lotteries, games, investment apps). All contracts that require some amount of money to participate have the potential to have that money lost in the contract if things don't go accordingly. Below is the $add\_player$ function from our RPS contract. The function adds a player and stores their unique identifier ($msg.sender$). The contract also takes a value ($msg.value$) that is sent to the contract. The value is the currency used by Ethereum, ether. Ether can be thought of as similar to bitcoins. Bitcoins are generated by mining, and can be used for trading and to pay transaction fees; ether is also mined, and is used as the currency to fuel all contracts as well as the currency that individuals will trade within contracts. Let's dive in and see if we can find a contract theft error in the $add\_player$ contract below: 
}
%\begin{mdframed}[leftmargin = -1cm, rightmargin = -1cm, linecolor=black, topline=true, bottomline=true,
  %leftline=false, rightline=false, backgroundcolor=lightgray!40]
%\begin{minted}
%[
%frame=lines,
%framesep=2mm,
%baselinestretch=1.2,
%fontsize=\footnotesize,
%linenos
%]
%{python}
\begin{figure}
\begin{mdframed}
\begin{minted}[breaklines, tabsize=2, fontsize=\footnotesize, linenos, xleftmargin=10pt]{python}
#adds players who send 1000 wei to the contract to the game
def add_player(choice):
	if num_players < 2 and msg.value == 1000:
		reward = reward + msg.value
		player[num_players].address = msg.sender
		player[num_players].choice = choice
		num_players = num_players + 1
		return(0)
	else:
		return(-1)
def check():
	p0_choice = player[0].choice
	p1_choice = player[1].choice
	#If player 0 wins
	if check_winner[p0_choice][p1_choice] == 0:
		send(0,player[0].address, reward)
		return(0)
	#If player 1 wins
	elif check_winner[p0_choice][p1_choice] == 1:
		send(0,player[1].address, reward)
		return(1)
	#If no one wins
	else:
		send(0,player[0].address, reward/2)
		send(0,player[1].address, reward/2)
		return(2)
\end{minted}
\end{mdframed}
\caption{
\label{fig:moneyleaks}
{\bf Pitfalls of contracts.} This buggy contract illustrates several pitfalls.
1) Line  When a third player 
sends inputs to the contract, the
money effectively sinks into a blackhole.
2) 
3) Line Players also send their inputs 
in the clear to the contract. A malicious player 
can wait to see the other player's 
choice before deciding its own.
Cryptographic commitments can remedy this problem.
To generalize, smart contracts programs
typically involve complex state machines, and
students often make mistakes encoding 
state machines correctly. 
%Typically, students
%left out corner cases of state machines,  
}
\end{figure}

\ignore{
In this section, a user adds themselves to the game by sending a small amount of ether with their transaction. The contract takes this ether, stored in $msg.value$, and adds it to the winnings pool, the prize that the winner of each round will receive. Let's consider two scenarios our contract currently allows 1) a potential entrant sends too much or too little ether, 2) there are already two participants, so additional players send transactions to join, but are not allowed. In both of the following scenarios the contract will keep their money. If someone sent too much or too little to enter they will not be added as a player, but their funds will be kept. Even worse, if the match is full any person who tries to join (they have no way of knowing it is full) will pay to play but never be added to a game! Both of these errors will cause distrust in our contract, eventually resulting in the community not trusting this particular contract and, more importantly, this contract's author - you.
}

%\elaine{work on the following text.}
\ignore{
So how do we fix these issues? It seems like our contract needs the ability to give refunds to users who try to sign up too late. Think about how you would do this. Go ahead and try it and see if your idea works! Are there any other edge cases where issuing a refund should be considered? Look at the section "Sending Wei" in the Serpent Tutorial for inspiration.
}

\subsection{Failure to Use Cryptography}
Players in a smart contract are selfish and  
may deviate from the ``honest'' behavior
to maximize their financial gains.
Cryptography is a means to provide 
defense against potentially malicious players.

In our smart contract programming lab, 
several project groups
neglected to use cryptography 
to provide proper defense initially.
Let us look at our ``Rock, Paper, Scissors'' example.
In Figure~\ref{fig:nocrypto}, players send their
choices of rock, paper, or scissors  
to the smart contract.
In the students' buggy programs, players
send their inputs in the clear.  
This neglects the fact that all messages and transactions
are broadcast 
across the entire cryptocurrency network, and observable
by every other player. 
A ``rational'' player would therefore wait
for the other player to send its inputs, before
deciding its own.


The obvious remedy to this problem is to use
a cryptographic commitment.
Both players can commit to their inputs in one time epoch,
and then open the commitments and reveal their inputs in a later epoch. 
A commitment satisfies two properties, {\it binding} and {\it hiding}.
Binding ensures that a player cannot change its input
after committing to it. Hiding ensures
that by observing the other player's commitment, a party learns
nothing about its input choice.  

In our undergraduate security class, commitment was originally not
part of the syllabus. The instructor
therefore used this as an opportunity to teach cryptographic commitments
to the class.
In the amendment phase of the project,
students were able to implement cryptographic commitments
to secure their smart contracts!
%therefore understood the motivation of cryptographic
%commitments  

\ignore{
Cryptography is often the first line of defense against security hazards in smart contract programming. In the example above, players reveal too much plaintext information, which can be used by an attacker to spoil the game. In the section, we'll describe how to apply cryptographic commitments to fix this problem.

In our RPS contract the user is using a numeric scale as their input with 0: rock, 1: paper, 2: scissors. Let's take a look at the function that registers their inputs and think about possible vulnerabilities:
}

%\begin{mdframed}[leftmargin = -1cm, rightmargin = -1cm, linecolor=black, topline=true, bottomline=true,
  %leftline=false, rightline=false, backgroundcolor=lightgray!40]
%\begin{minted}
%[
%frame=lines,
%framesep=2mm,
%baselinestretch=1.2,
%fontsize=\footnotesize,
%linenos
%]
%{python}

\elaine{in figures, can we put in framed boxes the lines
that have problems.}

\begin{figure}
\begin{mdframed}
\begin{minted}[breaklines, tabsize=2, fontsize=\footnotesize, linenos, xleftmargin=10pt]{python}
def add_player(player_commitment):
	if num_players < 2 and msg.value >= 1000:
		reward = reward + msg.value
		player[num_players].address = msg.sender
		player[num_players].commit = player_commitment
		num_players = num_players + 1
		if msg.value - 1000 > 0:
			send(0, msg.sender, msg.value-1000)
		return(0)
	else:	
		if msg.value > 0 :
			# prevent unnecessary leakage of money
			send(0, msg.sender, msg.value)
		return(-1)
		
#verifies the choice in their committed answer matches
def open(choice, nonce):
	#Ensure two players are in the contract
	if not num_players == 2: return(-1)
	#Determine which player submitted the open request
	if msg.sender == player[0].address:
		player_num = 0
	elif msg.sender == player[1].address:
		player_num = 1
	else:
		return(-1)
	#Check the commitment and ensure they have not tried to commit already
	if sha3([msg.sender, choice, nonce], items=3) == player[player_num].commit and not player[player_num].has_revealed:
		#If commitment verified, we should store choice in plain text
		player[player_num].choice = choice
		#Store current block number to give other player 10 blocks to open their commit
		player[player_num].has_revealed = 1		
		if not timer_start:
			timer_start = block.number
		return(0)
	else:
		return(-1)
\end{minted}
\end{mdframed}
\caption{
\label{fig:nocrypto}
{\bf Improved contract using refunds and commitments.}
}
\end{figure}

\ignore{
We can see that our $input()$ function identifies the sender with $msg.sender$ and then stores their input $choice$ in plaintext (where $choice$ = 0, 1, or 2). The lack of encryption means that the other player could see what their opponent played by looking at a block that published it; with that information they could input the winning choice to ensure they always win the prize pool. This can be fixed by using a commitment scheme. We will alter $input()$ to accept a hash of [sender, choice, and a nonce]. After both players have committed their inputs they will send their $choice$ and $nonce$ (as plaintext) to an $open()$ function. $open()$ will verify what they sent to $input()$. What they send to $open()$ will be hashed, and that hash will be checked against the hash the user committed through $input()$. If the two hashes don't match then the player will automatically lose based on the assumption they were being dishonest. Understanding where crypto elements should be used is crucial to justifying why others should use your contract. 

In order to enhance the security and fairness of our contract we will implement a commitment scheme using the hashing functions discussed earlier in this guide. The first change that is necessary in our contract is to have the $input()$ function accept the hash given from the user. Our RPS application would prompt the participants in our game to send a hash of their input and a nonce of their choosing. Thus $choice$ = SHA3(msg.sender's public address, numerical input (0 or 1 or 2) + $nonce$). This hashed value is stored in the contract, but there is no way for either opponent to discover the other's input based on their committed choice alone.\\

Now that we have the hash stored in the contract we need to implement an $open()$ function that we discussed earlier. Our $open()$ function will take the plaintext inputs and nonces from the players as parameters. We will hash these together with the unique sender ID and compare to the stored hash to verify that they claim to have committed as their input is true. Remember, up until this point the contract has \textit{no way of knowing} who the winner is because it has \textit{no way of knowing} what the inputs are. The contract doesn't know the nonce, so it cannot understand what the $choice$ sent to $input()$ was. Below is the updated, cleaned up contract (version2.py) implementing an $open()$ and modifying $check()$ to work with our new scheme. Notice we have added a method $open()$ and reorganized our $check()$:
}
%\begin{mdframed}[leftmargin = -1cm, rightmargin = -1cm, linecolor=black, topline=true, bottomline=true,
  %leftline=false, rightline=false, backgroundcolor=lightgray!40]
%\begin{minted}
%[
%frame=lines,
%framesep=2mm,
%baselinestretch=1.2,
%fontsize=\footnotesize,
%linenos
%]
%{python}
\begin{figure}
\begin{mdframed}
\begin{minted}[breaklines, tabsize=2, fontsize=\footnotesize, linenos, xleftmargin=10pt]{python}
#verifies the choice in their committed answer matches
def check():
	#Check to make sure at least 10 blocks have been given for both players to reveal their play.
	if block.number - timer_start < 10: return(-2)
	#check to see if both players have revealed answer
	if player[0].has_revealed and player[1].has_revealed:
		p0_choice = player[0].choice
		p1_choice = player[1].choice
		#If player 0 wins
		if check_winner[p0_choice][p1_choice] == 0:
			send(0,player[0].address, reward)
			return(0)
		#If player 1 wins
		elif check_winner[p0_choice][p1_choice] == 1:
			send(0,player[1].address, reward)
			return(1)
		#If no one wins
		else:
			send(0,player[0].address, reward/2)
			send(0,player[1].address, reward/2)
			return(2)
	#if p1 revealed but p2 did not, send money to p1
	elif player[0].has_revealed and not player[1].has_revealed:
		send(0,player[0].address, reward)
		return(0)
	#if p2 revealed but p1 did not, send money to p2
	elif not player[0].has_revealed and player[1].has_revealed:
		send(0,player[1].address, reward)
		return(1)
	#if neither p1 nor p2 revealed, keep both of their bets
	else:
		return(-1)
\end{minted}
\end{mdframed}
\caption{
\label{fig:crypto1}
{\bf Avoiding pitfalls using refunds and commitments.} This snippet corrects flaws in the initial version (Figure~\ref{fig:nocrypto})
}
\end{figure}

\subsection{Failure of Incentive Compatability}
More bugs remain even when the ones mentioned in Sections \elaine{refer}
are dealt with!
Suppose that now the programmer
did use cryptographic commitments to secure their 
``Rock, Paper, Scissors'' contract. 
What else can go wrong at this point?
This leads to the third class of typical mistakes
students made.

In the buggy contract snippet depicted 
in Figure~\ref{fig:incentive}, a party can 
wait for the other to open its commitment. Upon
seeing that it will lose, the party may elect
to abort -- thus denying payment
to the other player as well. 

This generalizes to a broader question of how to 
ensure the incentive compatibility of a contract.
In this specific example, we can remedy the problem
by making a player lose its deposit unless it opens
its commitment in a timely manner. 

\elaine{expand the discussion, discuss more generally what the incentive compatibility
problem is here.}


\ignore{
Designing an effective smart contract often means considering the incentives of the players involved, and aligning these incentives with the desired behavior. Can a user profit by using the contract in an unexpected way? Is ``honest'' behavior more expensive than the alternative?  We strive to make ``incentive compatible'' contracts, which roughly means that using the contract as intended is the most cost-effective behavior. In a typical escrow contract, a collateral deposit is collected from both individuals so they each have an incentive to complete their exchange. In a game contract where inputs are encrypted, a collateral deposit should be implemented to encourage both players to decrypt their responses within a time frame to avoid cheating or stalling the contract.  Let's look and see how our RPS contract holds up with regard to incentives:
}
%\begin{mdframed}[leftmargin = -1cm, rightmargin = -1cm, linecolor=black, topline=true, bottomline=true,
  %leftline=false, rightline=false, backgroundcolor=lightgray!40]
%\begin{minted}
%[
%frame=lines,
%framesep=2mm,
%baselinestretch=1.2,
%fontsize=\footnotesize,
%linenos
%]
%{python}
%\caption{
%\label{fig:incentive}
%{\bf Pitfall: incentive compatibility.}
%In this buggy contract snippet, when one player opens its commitment,
%the other play may elect to abort the protocol upon seeing that it will lose.
%To make the contract ``safe'', we can impose an incentive structure
%such that a player loses its deposit on failing to open
%its commitment within a certain timeout.  
%}
%\end{figure}

\ignore{
Given the version at the end of this section, our contract is \textit{almost} incentive compatible. Only one party needs to call the $check()$ function in order for the winnings to be fairly distributed to the actual winner, regardless of who calls. This requires one player to spend gas to check to see who won, while the other player doesn't need to spend any gas. There is currently no way to require two people to spend equal amount of gas to call one function. How could this affect the incentives of the contract? \\

In the next section we will look at how the current block number and the amount of blocks that have arrived previously affect the security of a contract. We will look to alter our contract further so that if someone doesn't open (verify) their rock/paper/scissors commitments within a given timeframe (i.e. 5 blocks after they are added to the contract), then the contract would send the money to the person who \textit{did} verify their input by the deadline. This incentivizes both users to verify their inputs before the $check()$ function is called after a random amount of blocks have been published. If you don't reveal your commitment, then you are \textit{guaranteed} to lose.
}

% \subsection{Further Paradigms of Contract Design}

\subsection{Original Buggy Rock, Paper, Scissor Contract}

%\begin{mdframed}[rightmargin = -1cm, leftmargin = -1cm, linecolor=black, topline=true, bottomline=true,
  %leftline=false, rightline=false, backgroundcolor=lightgray!40]
%\begin{minted}
%[
%frame=lines,
%framesep=2mm,
%baselinestretch=1.2,
%fontsize=\footnotesize,
%linenos
%]
%{python}
\begin{mdframed}
\begin{minted}[breaklines, tabsize=2, fontsize=\footnotesize, linenos, xleftmargin=10pt]{python}
data player[2](address, commit, choice, has_revealed)
data num_players
data reward
data timer_start #Only used in the fixed version of add_player, open and check.
data check_winner[3][3]

def init():
	#If 2, return tie (2)
	#If 0, player 0 wins
	#If 1, player 1 wins
	#0 = rock
	#1 = paper
	#2 = scissors
	#Tie
	check_winner[0][0] = 2
	check_winner[1][1] = 2
	check_winner[2][2] = 2
	#Rock beats scissors
	check_winner[0][2] = 0
	check_winner[2][0] = 1
	#Scissors beats paper
	check_winner[2][1] = 0
	check_winner[1][2] = 1
	#Paper beats rock
	check_winner[1][0] = 0
	check_winner[0][1] = 1
	#Track number of players in game
	num_players = 0
def add_player():
	#refer to figure 2
def open():
	#refer to figure 2
def check():
	#refer to figure 3
\end{minted}
\end{mdframed}

\subsection{Fixed ``Rock, Paper, Scissors'' Contract} 
The full, fixed ``Rock, Paper, Scissors'' contract
is included in our open-source lab materials
available at 
\elaine{fill in anonymous url}.

\section{Conclusion}
\subsection{Open-Source Course and Lab Materials}
%Besides the aforementioned 
%pitfalls of smart contract programming, 
Our smart contract programming lab was an audacious 
intial  attempt 
at instructing a technology of in-development nature.
%We also drew several lessons 
%that will be useful for the future. 
%through 
%this smart contract programming lab.
Ethereum and its Serpent language
have only recently emerged, and are 
rapidly undergoing changes. 
The Serpent language is not well documented and development
environment support (e.g., debugging tools) 
is also rudimentary.
Therefore, several students struggled 
in installing the simulation environment and 
getting up to speed.

To faciliate future pedogogical endeavors on smart contract programming,
and avoid issues resulting from the in-development nature of the technology, 
%repeating some of the aforementioned issues, 
we will shortly open source well-structured course and lab materials on
smart contract programming.
For a sneak peek of our course 
materials, please visit \elaine{fill in anonymous url}.

The course materials include the following:
\begin{itemize}[leftmargin=5mm]
\item
A detailed language reference 
guide for Ethereum's Serpent language -- an example
smart contract language adopted in the lab.
\item
A virtual machine image with a snapshop of {\tt pyethereum} installed,
providing a simulator environment for experimentation.
Since the Ethereum's Serpent language is constantly
under development, our Serpent language
reference matches with the snapshot installed in this VM. 
\item
A tutorial that 
builds on our ``Rock, Paper, Scissors'' example, 
intended to 
walk the student through the typical pitfalls
in programming safe smart contracts.
The student is presented with the buggy version of the contract
and asked to fix the bugs in a step-by-step, guided manner.
\end{itemize}


\subsection{Cryptocurrency and Smart Contracts as 
a Cybersecurity Pedagogical Platform}
Our experiences also led us to conclude 
that cryptocurrency and smart contracts are a
great platform for cybersecurity pedagogy. 
First, cryptocurrency and smart contracts, like
other cool emerging technologies, could easily capture the students'
attention and imagination.
Second, 
cybersecurity is a science that is interdisciplinary in nature;
and cryptocurrency is a platform that captures 
multiple core cybersecurity notion, e.g., cryptography, 
programming languages, and incentives. 
Third, cryptocurrency and smart contracts
easily motivate ``adversarial thinking''. For example,
in our lab, students had to analyze their own smart contracts
and reason how other selfish players can harm 
honest participants and maximize their own financial gains.

\ignore{
in our lab, students were able to compose smart contracts,
and then conduct in-depth security analyses of their own smart contracts.
In this process, students learned the 
essense of ``adversarial thinking'', e.g., by reasoning how
selfish participants in a smart contract 
can maximize its financial gains and harm honest players. 
}


\subsection{The ``Build, Break, and Amend Your Own Programs''
Approach to Cybersecurity Education}
Inspired by our smart contract programming lab, 
we also felt that the 
``Build, break, and amend your own programs''
is very helpful for cybersecurity education. 

In our labs, students learned why security is difficult 
and learned adversarial thinking
by analyzing and breaking 
their own programs.  
Students initially  
failed to make proper use of cyrptography in their 
smart contracts (see Section \elaine{refer back}).
But then, by realizing
why their smart contracts are not safe, they 
become self-driven in learning cryptographic building blocks.
%-- then, by
%conducting a security analysis of their own contracts,
%students learned why cryptography is cool and 
%this in turn motivated their learning of cryptographic topics.

In future work, we plan to further extend these pedagogical ideas, 
such that students can learn through hands-on, 
creative experiences, and learn adversarial thinking 
through attacking and amending their own code.



% \begin{thebibliography}{9}

% \bibitem{Using pyethereum.tester}
% 	Using pyethereum.tester. Pyethereum Github. 2014. \url{https://github.com/ethereum/pyethereum/wiki/Using-pyethereum.tester}

% \bibitem{test_contracts.py}
% 	pyethereum/tests/test\_contracts.py. Pyethereum Github. 2015. \url{https://github.com/ethereum/pyethereum/blob/develop/tests/test_contracts.py}

% \bibitem{Serpent}
% 	Serpent. Ethereum Wiki. 2015. \url{https://github.com/ethereum/wiki/wiki/Serpent}

% \bibitem{Serpent 1.0 (old)}
% 	Serpent 1.0 (old). Ethereum Wiki. 2015. \url{https://github.com/ethereum/wiki/wiki/Serpent-1.0-(old)}

% \bibitem{PeterBorah 2014}
% 	PeterBorah. ethereum-powerball. 2014. \url{https://github.com/PeterBorah/ethereum-powerball/tree/master/contracts}

% \bibitem{KenK's First Contract Tutorial}
% 	KenK. Dec. 2014. \url{http://forum.ethereum.org/discussion/1634/tutorial-1-your-first-contract}

% \bibitem{Shi 2015}
% 	Shi, E. Undergraduate Ethereum Lab at Maryland and Insights Gained. 2015. \url{https://docs.google.com/presentation/d/1esw_lizWG06zrWaOQKcbwrySM4K9KzmRD3rtBUx0zEw/edit?usp=sharing}

% \bibitem{Ethereum White Paper}
% 	Buterin, V. 2014. \url{https://www.ethereum.org/pdfs/EthereumWhitePaper.pdf}

% \end{thebibliography}

\bibliographystyle{plain}
\bibliography{serpent_bib}

\end{document}
