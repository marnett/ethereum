\documentclass[12pt]{article}
\usepackage{fullpage}
\usepackage{listings}
\usepackage{framed}
\usepackage[usenames,dvipsnames,svgnames,table]{xcolor}
\usepackage{mdframed}
\usepackage{minted}
\PassOptionsToPackage{hyphens}{url}\usepackage{hyperref}

\lstset{
	tabsize = 4
}

\begin{document}
\title{Lab: Step by Step towards Programming a Safe Smart Contract}

\author{
  Kevin Delmolino\\
  \texttt{del@terpmail.umd.edu}
  \and
  Mitchell Arnett\\
  \texttt{marnett@umd.edu}
  \and
  Ahmed Kosba\\
  \texttt{akosba@cs.umd.edu}
  \and
  Andrew Miller\\
  \texttt{amiller@cs.umd.edu}
  \and
  Elaine Shi\\
  \texttt{elaine@cs.umd.edu}
}

\maketitle

\setcounter{tocdepth}{5}
\tableofcontents

\newpage
\section{Introduction}

\paragraph{Intended audience.}
This tutorial is intended for instructors
who wish to conduct a smart  
contract programming lab, or students/developers
who want to learn about smart contract programming.


\paragraph{Smart contract programming model.}
While we use Ethereum's Serpent language as a specific example,
this document is 
intended to be broadly applicable to 
smart contract programming on top of a decentralized cryptocurrency.

Loosely speaking, we shall make some simplifying assumptions 
about the security model. 
We assume that the decentralized
cryptocurrency under consideration has a secure and incentive compatible
consensus protocol, such that when
rational miners comprise the majority of compute
power (or other forms of resources),
in a Nash equilibrium, it is in the best interest
of rational miners to honestly execute a
contract's program logic.
In reality, existing decentralized cryptocurrency
achieves only heuristic security. But we will make this assumption
nevertheless. How to design a provably  
secure decentralized consensus protocol under
rationality assumptions is a topic of 
future research.


\paragraph{Accompanying materials.}
All example smart contracts 
in this document are written for a specific 
snapshot of Ethereum.  
For the users' convenience, we offer
a VM image with appropriate  
versions of the software pre-installed~\cite{vmimage}.
We also provide detailed Ethereum reference manuals
geared towards this specific 
snapshot of Serpent~\cite{serpentref}.
Finally, we also recommend the reader
to a more concise, Powerpoint presentation of this tutorial
by Elaine Shi and Andrew Miller~\cite{Shi2015}.


For best learning outcomes, 
we strongly recommend that the reader take a hands-on approach
towards reading this lab tutorial, by running
our examples in {\tt py-ethereum}, and 
deploying fixes to the 
pitfalls that we point out in this document.



\section{A Rock-Paper-Scissors Example}
%Now that we have gone through and annotated several contract examples it is time to consider a couple key design concepts required to create a high-level smart contract. 
%By the end of this section we will talk about several key mistakes that show up in high-level contracts, and you will aim to identify and resolve them in a rock, paper, scissor contract example (RPS).
We use a very simple Rock-Paper-Scissors contract to illustrate
several pitfalls of smart contract programming.
Mistakes resembling these 
were actually observed in our Smart Contract Programming Lab  
in ``CMSC 414 - Undergraduate Security''.


\subsection{Corner Cases in Coding State Machines}
The first contract design error we will talk about is contracts causing money to disappear. Some contracts require the participants to send an amount of money to enter the contract (lotteries, games, investment apps). All contracts that require some amount of money to participate have the potential to have that money lost in the contract if things don't go accordingly. Below is the $add\_player$ function from our RPS contract. The function adds a player and stores their unique identifier ($msg.sender$). The contract also takes a value ($msg.value$) that is sent to the contract. The value is the currency used by ethereum, Ether. Ether can be thought of in a similar light to Bitcoin; Ether is mined and used as the currency to fuel all contracts as well as the currency that individuals will trade within contracts. Let's dive in and see if we can find a contract theft error in the $add\_player$ contract below: 

\begin{mdframed}[leftmargin = -1cm, rightmargin = -1cm, linecolor=black, topline=true, bottomline=true,
  leftline=false, rightline=false, backgroundcolor=lightgray!40]
\begin{minted}
[
frame=lines,
framesep=2mm,
baselinestretch=1.2,
fontsize=\footnotesize,
linenos
]
{python}
def add_player():
	if not self.storage["player1"]:
		if msg.value == 1000:
			self.storage["WINNINGS"] = 
				self.storage["WINNINGS"] + msg.value
			self.storage["player1"] = msg.sender
			return(1)
		return (0)
	elif not self.storage["player2"]:
		if msg.value == 1000:
			self.storage["WINNINGS"] = 
				self.storage["WINNINGS"] + msg.value
			self.storage["player2"] = msg.sender
			return(2)
		return (0)
	else:
		return(0)
\end{minted}
\end{mdframed}

In this section a user adds themselves to the game by sending a small amount of Ether with their transaction. The contract takes this Ether, stored in $msg.value$, and adds it to the winnings pool, the prize that the winner of each round will receive. Let's consider two scenarios our contract currently allows 1) a potential entrant sends too much or too little Ether, 2) there are already two participants, so additional players send transactions to join, but are not allowed. In both of the following scenarios the contract will keep their money. If someone sent too much or too little to enter they will not be added as a player, but their funds will be kept. Even worse, if the match is full any person who tries to join (they have no way of knowing it is full) will pay to play but never be added to a game! Both of these errors will cause distrust in our contract, eventually resulting in the community not trusting this particular contract and, more importantly, this contract's author - you.

So how do we fix these issues? It seems like our contract needs the ability to refund - think about how you would do this. Go ahead and try it and see if your idea works! Are there any other edge cases where issuing a refund should be considered? Look at the previous section "Sending Wei" for more information. 

\subsection{Implementing Cryptography}
It goes without saying that as a student in a computer security course you would implement cryptographic practices wherever you can. Thus given a contract that requires impactful user inputs ( ones that affect the outcome of said contract) cryptography should be implemented. In our RPS contract the user is using a numeric scale as their input with 0: rock, 1: paper, 2: scissors. Let's take a look at the function that registers their inputs and think about possible vulnerabilities:

\begin{mdframed}[leftmargin = -1cm, rightmargin = -1cm, linecolor=black, topline=true, bottomline=true,
  leftline=false, rightline=false, backgroundcolor=lightgray!40]
\begin{minted}
[
frame=lines,
framesep=2mm,
baselinestretch=1.2,
fontsize=\footnotesize,
linenos
]
{python}
def input(choice):
	if self.storage["player1"] == msg.sender:
		self.storage["p1value"] = choice
		return(1)
	elif self.storage["player2"] ==  msg.sender:
		self.storage["p2value"] = choice
		return(2)
	else:
		return(0)
\end{minted}
\end{mdframed}

We can see that our $input()$ function identifies the sender with $msg.sender$ and then stores their input $choice$ in plaintext (where $choice$ = 0, 1, or 2). The lack of encryption means that the other player could see what their opponent played by looking at a block that published it; with that information they could input the winning choice to ensure they always win the prize pool. This can be fixed by using a commitment scheme. We will alter $input()$ to accept a hash of [sender, choice, and a nonce]. After both players have committed their inputs they will send their $choice$ and $nonce$ (as plaintext) to an $open()$ function. $open()$ will verify what they sent to $input()$. What they send to $open()$ will be hashed, and that hash will be checked against the hash the user committed through $input()$. If the two hashes don't match then the player will automatically lose based on the assumption they were being dishonest. Understanding where crypto elements should be used is crucial to justifying why others should use your contract. 

In order to enhance the security and fairness of our contract we will implement a commitment scheme using the hashing functions discussed earlier in this guide. The first change that is necessary in our contract is to have the $input()$ function accept the hash given from the user. Our RPS application would prompt the participants in our game to send a hash of their input and a nonce of their choosing. Thus $choice$ = SHA3(msg.sender's public address, numerical input (0 or 1 or 2) + $nonce$). This hashed value is stored in the contract, but there is no way for either opponent to discover the other's input based on their committed choice alone.\\

Now that we have the hash stored in the contract we need to implement an $open()$ function that we discussed earlier. Our $open()$ function will take the plaintext inputs and nonces from the players as parameters. We will hash these together with the unique sender ID and compare to the stored hash to verify that they claim to have committed as their input is true. Remember, up until this point the contract has \textit{no way of knowing} who the winner is because it has \textit{no way of knowing} what the inputs are. The contract doesn't know the nonce, so it cannot understand what the $choice$ sent to $input()$ was. Below is the updated, cleaned up contract (version2.py) implementing an $open()$ and modifying $check()$ to work with our new scheme. Notice we have added a method $open()$ and reorganized our $check()$:

\begin{mdframed}[leftmargin = -1cm, rightmargin = -1cm, linecolor=black, topline=true, bottomline=true,
  leftline=false, rightline=false, backgroundcolor=lightgray!40]
\begin{minted}
[
frame=lines,
framesep=2mm,
baselinestretch=1.2,
fontsize=\footnotesize,
linenos
]
{python}
def input(player_commitment):
	if self.storage["player1"] == msg.sender:
		self.storage["p1commit"] = player_commitment
		return (1)
	elif self.storage["player2"] ==  msg.sender:
		self.storage["p2commit"] = player_commitment
		return(2)
	else:
		return(0)

def open(choice, nonce):
	if self.storage["player1"] == msg.sender:
		if sha3([msg.sender, choice, nonce], items=3) == self.storage["p1commit"]:
			self.storage["p1value"] = choice
			self.storage["p1reveal"] = 1
			return(1)
		else:
			return(0)
	elif self.storage["player2"] == msg.sender:
		if sha3([msg.sender, choice, nonce], items=3) == self.storage["p2commit"]:
			self.storage["p2value"] = choice
			self.storage["p2reveal"] = 1
			return(2)
		else:
			return(0)
	else:
		return(-1)

def check():
	#check to see if both players have revealed answer
	if self.storage["p1reveal"] == 1 and self.storage["p2reveal"] == 1:
		#If player 1 wins
		if self.winnings_table[self.storage["p1value"]][self.storage["p2value"]] == 1:
			send(100,self.storage["player1"], self.storage["WINNINGS"])
			return(1)
		#If player 2 wins
		elif self.winnings_table[self.storage["p1value"]][self.storage["p2value"]] == 2:
			send(100,self.storage["player2"], self.storage["WINNINGS"])
			return(2)
		#If no one wins
		else:
			send(100,self.storage["player1"], 1000)
			send(100,self.storage["player2"], 1000)
			return(0)
	#if p1 revealed but p2 did not, send money to p1
	elif self.storage["p1reveal"] == 1 and not self.storage["p2reveal"] == 1:
		send(100,self.storage["player1"], self.storage["WINNINGS"])
		return(1)
	#if p2 revealed but p1 did not, send money to p2
	elif not self.storage["p1reveal"] == 1 and self.storage["p2reveal"] == 1:
		send(100,self.storage["player2"], self.storage["WINNINGS"])
		return(2)
	#if neither p1 nor p2 revealed, keep both of their bets
	else:
		return(-1)
\end{minted}
\end{mdframed}

\subsection{Incentive Compatability}
The final key bug to watch out for is incentive incompatibility. There are contract ideas that must consider user incentives in order for them to run as planned. If I had an escrow contract incentives must be implemented so both individuals don't always so they did not receive their promised service. If I have a game contract where inputs are encrypted, incentives must be implemented to ensure both players decrypt their responses within a time frame to avoid cheating. Let's look and see how our RPS contract holds up with regard to incentives:

\begin{mdframed}[leftmargin = -1cm, rightmargin = -1cm, linecolor=black, topline=true, bottomline=true,
  leftline=false, rightline=false, backgroundcolor=lightgray!40]
\begin{minted}
[
frame=lines,
framesep=2mm,
baselinestretch=1.2,
fontsize=\footnotesize,
linenos
]
{python}
def check():
	#check to see if both players have revealed answer
	if self.storage["p1reveal"] == 1 and self.storage["p2reveal"] == 1:
		#If player 1 wins
		if self.winnings_table[self.storage["p1value"]][self.storage["p2value"]] == 1:
			send(100,self.storage["player1"], self.storage["WINNINGS"])
			return(1)
		#If player 2 wins
		elif self.winnings_table[self.storage["p1value"]][self.storage["p2value"]] == 2:
			send(100,self.storage["player2"], self.storage["WINNINGS"])
			return(2)
		#If no one wins
		else:
			send(100,self.storage["player1"], 1000)
			send(100,self.storage["player2"], 1000)
			return(0)
	#if p1 revealed but p2 did not, send money to p1
	elif self.storage["p1reveal"] == 1 and not self.storage["p2reveal"] == 1:
		send(100,self.storage["player1"], self.storage["WINNINGS"])
		return(1)
	#if p2 revealed but p1 did not, send money to p2
	elif not self.storage["p1reveal"] == 1 and self.storage["p2reveal"] == 1:
		send(100,self.storage["player2"], self.storage["WINNINGS"])
		return(2)
	#if neither p1 nor p2 revealed, keep both of their bets
	else:
		return(-1)
\end{minted}
\end{mdframed}

Given the version at the end of this section our contract is \textit{almost} incentive compatible. Only one party needs to call the $check()$ function in order for the winnings to be fairly distributed to the actual winner, regardless of who calls. This requires one player to spend gas to check to see who won, while the other player doesn't need to spend the same amount. There is currently no way to require two people to spend equal amount of gas to call one function. How could this affect the incentives of the contract? \\

In the next section we will look at how the current block number and the amount of blocks that have passed affect the security of a contract. We will look to alter our contract further so if someone doesn't open (verify) their rock/paper/scissors within a given timeframe (i.e. 5 blocks after they are added to the contract), the contract would, by default, send the money to the person who \textit{did} verify their input by the deadline. This incentivizes both users to verify their inputs before the $check()$ function is called after a random amount of blocks have been published; if you don't verify you are \textit{guaranteed} to lose.

% \subsection{Further Paradigms of Contract Design}

\subsection{Original Buggy Rock, Paper, Scissor Contract}

\begin{mdframed}[rightmargin = -1cm, leftmargin = -1cm, linecolor=black, topline=true, bottomline=true,
  leftline=false, rightline=false, backgroundcolor=lightgray!40]
\begin{minted}
[
frame=lines,
framesep=2mm,
baselinestretch=1.2,
fontsize=\footnotesize,
linenos
]
{python}
data winnings_table[3][3]

def init():
	#If 0, tie
	#If 1, player 1 wins
	#If 2, player 2 wins

	#0 = rock
	#1 = paper
	#2 = scissors

	self.winnings_table[0][0] = 0
	self.winnings_table[1][1] = 0
	self.winnings_table[2][2] = 0

	#Rock beats scissors
	self.winnings_table[0][2] = 1
	self.winnings_table[2][0] = 2

	#Scissors beats paper
	self.winnings_table[2][1] = 1
	self.winnings_table[1][2] = 2

	#Paper beats rock
	self.winnings_table[1][0] = 1
	self.winnings_table[0][1] = 2

	self.storage["MAX_PLAYERS"] = 2
	self.storage["WINNINGS"] = 0

def add_player():
	if not self.storage["player1"]:
		if msg.value == 1000:
			self.storage["WINNINGS"] = self.storage["WINNINGS"] + msg.value
			self.storage["player1"] = msg.sender
			return(1)
		return (0)
	elif not self.storage["player2"]:
		if msg.value == 1000:
			self.storage["WINNINGS"] = self.storage["WINNINGS"] + msg.value
			self.storage["player2"] = msg.sender
			return(2)
		return (0)
	else:
		return(0)

def input(choice):
	if self.storage["player1"] == msg.sender:
		self.storage["p1value"] = choice
		return(1)
	elif self.storage["player2"] ==  msg.sender:
		self.storage["p2value"] = choice
		return(2)
	else:
		return(0)

def check():
	#If player 1 wins
	if self.winnings_table[self.storage["p1value"]][self.storage["p2value"]] == 1:
		send(100,self.storage["player1"], self.storage["WINNINGS"])
		return(1)
	#If player 2 wins
	elif self.winnings_table[self.storage["p1value"]][self.storage["p2value"]] == 2:
		send(100,self.storage["player2"], self.storage["WINNINGS"])
		return(2)
	#If no one wins
	else:
		send(100,self.storage["player1"], self.storage["WINNINGS"]/2)
		send(100,self.storage["player2"], self.storage["WINNINGS"]/2)
		return(0)

def balance_check():
	log(self.storage["player1"].balance)
	log(self.storage["player2"].balance)
\end{minted}
\end{mdframed}

\section{State Machine Transitions}

\paragraph{Maintaining State in Smart Contracts}

In many scenarios, there is a need to adapt the behavior of a contract depending on the the value/number of transactions it receives, or depending on how much time has passed since a certain event. In other words, several applications need a stateful contract that acts differently to similar messages, depending on its state.

Maintaining the notion of a state in a contract requires a mechanism to handle state transitions, which we classify into event-based and time-based. We present simple approaches for how to express these in Serpent.

\paragraph{Event-based state transitions}

In this case, the state changes based on messages or transactions that the contract receives. One example is a puzzle contract that gives a reward to the first one who solves the problem, or a game contract that waits for two players to join, and then starts the game. The behavior of the contract should adapt when such events happen, or otherwise money may be lost unnecessarily. Coding such contracts can be straightforward via maintaining state variables in the contract storage. The following shows a proof-of-work contract that gives an award to the first sender who solves a Bitcoin-like proof-of-work puzzle.

\begin{mdframed}[rightmargin = -1cm, leftmargin = -1cm, linecolor=black, topline=true, bottomline=true,
  leftline=false, rightline=false, backgroundcolor=lightgray!40]
\begin{minted}
[
frame=lines,
framesep=2mm,
baselinestretch=1.2,
fontsize=\footnotesize,
linenos
]
{python}
def init(puzzle, target):
     self.storage["isSolved"] = 0                 ## State variable
     self.storage["puzzle"] = puzzle
     self.storage["target"] = target           
 

def receiveSolution(solution):
    if(self.storage["isSolved"] == 0 AND 
    	SHA3([self.storage["puzzle"], solution],2) < self.storage["target"]):
send(msg.sender, 10000)             # Sending reward
self.storage["isSolved"] = 1   # Changing the state variable
\end{minted}
\end{mdframed}




Time-based state transitions: 

Employing event-based transitions may not be enough to capture all the possible scenarios in typical applications. Think of an auction that accepts any number of bidders, but it sets a specific time frame to run the bidding stage, after which no more bids are accepted. The contract in this case should have a way to decide whether to accept bids or not based on the time of the transaction.

Two sample ways of how to use time in the contract is via using block.timestamp or block.number. For example, this is a part of an auction contract that only accepts bids submitted before a deadline. The deadline is 100 blocks ahead from the contract creation time.

\begin{mdframed}[rightmargin = -1cm, leftmargin = -1cm, linecolor=black, topline=true, bottomline=true,
  leftline=false, rightline=false, backgroundcolor=lightgray!40]
\begin{minted}
[
frame=lines,
framesep=2mm,
baselinestretch=1.2,
fontsize=\footnotesize,
linenos
]
{python}
def init():
     self.storage["deadline"] = 100 + block.number                
     # Think of other auction details  

def receiveBid(bid):
     if(block.number <= self.storage["deadline"]):
                # accept bid
     else:
                # abort
\end{minted}
\end{mdframed}                

(Exercise: Think how to complete the auction contract above. You can add other methods.).

\paragraph{Hybrid cases}

Sometimes, complex contracts may need to incorporate both kinds of state transitions. For example, think of a fundraising contract that will directly end when a certain amount of money is collected, but in the same time, the contract will have to run in just one month. Therefore, the contract must change its state if a month passes, or when the contract balance exceeds a threshold, whatever happens first.\\

(Exercise: Think how to write a fundraising contract as described above).



% \begin{thebibliography}{9}

% \bibitem{Using pyethereum.tester}
% 	Using pyethereum.tester. Pyethereum Github. 2014. \url{https://github.com/ethereum/pyethereum/wiki/Using-pyethereum.tester}

% \bibitem{test_contracts.py}
% 	pyethereum/tests/test\_contracts.py. Pyethereum Github. 2015. \url{https://github.com/ethereum/pyethereum/blob/develop/tests/test_contracts.py}

% \bibitem{Serpent}
% 	Serpent. Ethereum Wiki. 2015. \url{https://github.com/ethereum/wiki/wiki/Serpent}

% \bibitem{Serpent 1.0 (old)}
% 	Serpent 1.0 (old). Ethereum Wiki. 2015. \url{https://github.com/ethereum/wiki/wiki/Serpent-1.0-(old)}

% \bibitem{PeterBorah 2014}
% 	PeterBorah. ethereum-powerball. 2014. \url{https://github.com/PeterBorah/ethereum-powerball/tree/master/contracts}

% \bibitem{KenK's First Contract Tutorial}
% 	KenK. Dec. 2014. \url{http://forum.ethereum.org/discussion/1634/tutorial-1-your-first-contract}

% \bibitem{Shi 2015}
% 	Shi, E. Undergraduate Ethereum Lab at Maryland and Insights Gained. 2015. \url{https://docs.google.com/presentation/d/1esw_lizWG06zrWaOQKcbwrySM4K9KzmRD3rtBUx0zEw/edit?usp=sharing}

% \bibitem{Ethereum White Paper}
% 	Buterin, V. 2014. \url{https://www.ethereum.org/pdfs/EthereumWhitePaper.pdf}

% \end{thebibliography}

\bibliographystyle{plain}
\bibliography{serpent_bib}

\end{document}
